\section{Determine if the finger is touching the paper}
It is difficult to determine whether the finger touches the paper solely with the image taken from the direct above. As such, we use a light source to project the shadow of the hand onto the side of the hand. Suppose the light source is set to have angle $\theta$ from the vertical line, and the fingertip is at height $h$, then the distance between the fingertip of the shadow and the x-coordinate of the real fingership is $d = h \tan(\theta)$. So, if $\theta$ is not $0$, we can find $h$ with $h = d \cot(\theta)$. Or, since we only care if the finger is touching the paper or not, we can set a threshold $d_{th}$ such that we can assume the finger is touching the paper if and only if $d < d_{th}$. 

This method thus requires the detection of the shadow and its fingertip. We detect the shadow with the similar method for hand detection: using color filters and HSV filters. The actually condition we use are as follows:
\begin{itemize}
\item
$R < 107.1 $\\
\item
$G < 86.7 $\\
\item
$B < 120.6 $\\
\item
$V < 100.8 $\\
\end{itemize}•
, where $R, G, B$ are the red, green and blue values respectively, and $V$ is the value in HSV representation. 

To find the fingertip of the shadow, we can not use the same method proposed for finding the real fingertip because a large portion of the shadow is blocked by the hand when the hand is close to the paper, thus the center of mass of the shadow can be greatly affected.  Instead, we simply assume that the fingertip is always pointing down, and find the pixel of the shadow that has the lowest y-coordinate. Because we set the ligth source to be at the lower left corner of the screen, the shadow of the hand will always be at the upper-right side of it. Under the assumption that the finger is always pointing down, we ignore all the shadow pixels that are to the left and lower than the fingertip to reduce  noise. Further more, since we only care if the fingertip is touching the paper, we can ignore all the shadow pixels that are too far away from the real fingertip. So, we only find the shadow fingertip in the region of $\{(x,y): x_{finger}<x<x_{finger}+40, y_{finger}<y\}$. 

Finally, we have to decide the threshold for the distance between the real fingertip and the shadow fingertip. This threshold is affected by the posture. When the user is posing the palm posture and is touching the paper, a larger portion of the shadow will be blocked than when the user is posing the pointing finger. So, the threshold we decided was 30 for the pointing finger and 50 for the palm. 
